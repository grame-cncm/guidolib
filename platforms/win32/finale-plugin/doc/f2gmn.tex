\documentclass{article}
\begin{document}
\title{Finale Plugin: Export to GUIDO Music Notation}
\author{by Tobias Mann}
\maketitle
\section{Introduction}
The "Export to GUIDO" plugin converts music written in Finale 2001b to
GUIDO Music Notation. This would normally be done by a stand-alone converter, but
Finale's file format is not publicly specified. Instead, a plugin SDK is provided
and documented.\\
This documentation, however, proved to be incomplete. It was my main source
of information anyway, along with the example files and the information
available at http://www.finaletips.nu. Some information was guessed and
deduced.

\section{Capabilities}
The plugin is able to write files in the advanced GUIDO format. It can work on
a music document as a whole or on a selection. A few options
are available about whether to include a certain kind of information.
\subsection{What it can export}
\begin{itemize}
\item Tone heights (for standard key signatures)
\item Durations (except for very unusual time signatures such as 37/19)
\item Tuplets with ratios that are multiples of 2:3 or 4:5
\item Rests (this deserves special mentioning, see below...) can be exported to
\verb"_" or \verb"empty", depending on the situation. 
\item Time and key signatures if recognized. Major/minor information is preserved.
\item Ties. The plugin represents them with \verb"\tieBegin:" and \verb"\tieEnd:", using
the midi tone height as a label.
\item Beams. This is limited to up to two levels of nested \verb"\beam" tags and does not
work correctly in complicated cases. Still, it should even then improve readability. 
\end{itemize}
Multiple layers are represented as segments that are optionally forced into one
line using the \verb"\staff" tag. Parallel voices forced into one layer are
split as well, as GUIDO knows no other suitable representation. This does of
course not apply to (a series of) chords.

\subsection{What it cannot export yet}
For example:
\begin{itemize}
\item Dynamics and articulation (except if one counts beaming): slurs, crescendo, ...
\item Repetition bars
\item Text: lyrics, instrument names, title, composer, ...
\item Cosmetic graphics: stem directions...
\end{itemize}

\section{A Rest's Tale}
I've discovered three different ways in which Finale represents rests.
\begin{itemize}
\item Rests may be represented as chords with zero notes. This is
straight-forward and logical, as Finale treats all notes as chords.
\item Rests may be represented by the ommitance of notes/chords at all.
This can occur for a measure that is still empty, but also if the notes'
offsets and durations leave a gap. For example, there might be two voices
in one single layer, one of them being incomplete.\\
This case has been the most difficult to implement. Until I discovered this
manner of rest representation, I thought a one-run export would be possible.
\item Rests may be single-note chords with a certain flag enabled. I do
not know the reason for this possibility being in existence.
\end{itemize}
I shall call these types chord-, gap- and flag-rests respectively.\\
Flag-rests are always exported to \verb"_". So are the other types of rests,
if they occur in the first used layer of a staff. Rests other than flag-rests
that occur in secondary layers will be exported to \verb"empty".\\
This seems to be a good approximation of Finale's behaviour (which has no real
\verb"empties" at it's disposal, though).

\section{How the plugin works}
\subsection{Overview}
Finale plugins are DLLs that have been renamed to bear the extension \verb".fxt".
As described in the PDK reference, each plugin must provide six entry points,
the one interesting being \verb"FinaleExtensionInvoke" (in our case).
These entry points make up the first part of the main source file.\\
\verb"FinaleExtionsionInvoke" first shows a dialog using \verb"FX_dialog". The
related message handler can be found at the end of the source file.\\
Using this dialog, the user can (de)select several options, which are stored
in the global flag field \verb"gmnFormat".\\
Then \verb"FinaleExtensionInvoke" asks for a destination file name. A proposal is
derived from the original file name which is read using \verb"GetWindowText".\\
If the operation is not canceled, \verb"FinaleExtensionInvoke" will call
\verb"readMusic". \verb"readMusic" builds the bulk of the plugin. It 
iterates through the selected part of music with a bunch of nested \verb"for"
loops.

\subsection{Processing order}
In general, output is written as soon as input is processed. As some
information must be moved into another context due to organisational differences
between Finale's data structures and GUIDO, two rearrangement approaches used.\\
Observe that Finale represents music mainly as a tree (that is linearized by
\verb"for" loops). On the other hand GUIDO uses a (linear) text string to
represent a four-level tree.
\begin{itemize}
\item In order to deal with gap-rests and implicit polyphony, input is rearranged
when necessary:
Whenever \verb"readMusic" encounters a note's time offset that indicates it
should already have been
written out, it marks the note to be processed later, when it will be put into
another voice. These "marks" build up a linked list called \verb"Storage", which
is filled when a layer is first read and emptied again in the follow-up. 
\item In order to make the conversion from flags to bracket constructions easier,
positions within the output stream can be marked for later insertion of text
(such making the addition of a begin-tag possible, for example). This is currently
only used for beaming, but capable of multiple labeled markers.
\end{itemize}

\subsection{How to use w2g}
\verb"w2g" is an instance of \verb"writeBufferClass" and the start of a linked list of
the same type. There cannot be other linked lists of this type because \verb"w2g"
depends on the static variable \verb"writeBufferClass::count", but additional lists
wouldn't be useful anyway.\\
\verb"w2g" provides the code to set multiple labeled markers that is mentioned above.
If you don't need this feature, just call \verb"w2g.write" to write the destination
file. Don't call \verb"fwrite" directly, as it would scramble output.\\
To create a marker, just select an unused positive integer value. You may want to
\verb"define" it next to \verb"M_BEAM". Then mark the current output-stream position
using \verb"w2g.mark". To insert at this position, use \verb"w2g.insert".
\verb"w2g.write" will always append to the output stream.\\
Use \verb"w2g.flush" to delete a marker or, if the second argument is omitted, all
markers. \verb"flush" and \verb"write" are the only methods that can actually write
to the destination file. Before closing the file, you should be sure the buffer's
empty by using \verb"flush" without a marker ID.

\subsection{Description of important functions and macros}
The following functions are listed as they appear in the source file.
\begin{description}
\item[FinaleExtensionEnumerate] creates all menu entries (that is one in total)
in Finale's plugin menu.
\item[getDefaultGmnFilename] uses GetWindowText to create a default destination
file name from the original file name that is shown at Finale's title bar.
\item[FinaleExtensionInvoke] shows the options dialog and calls
\verb"readMusic" with the selection to process.
\item[writeBufferClass::add] appends to the linked buffer list.
\item[writeBufferClass::removeMark] removes one or all markers.
\item[writeBufferClass::flush] Calls \verb"removeMark" and then flushes the linked list up
to the next marker.
\item[writeBufferClass::write] Writes to the destination file or appends to the linked buffer
list, if necessary.
\item[writeBufferClass::mark] sets a marker.
\item[writeBufferClass::insert] inserts at a marked position.
\item[Storage::storageDone] Tells if there are unhandled notes in the
storage queue (see processing order above).
\item[store] Stores a note for later processing.
\item[stored] Checks if a given note is in the storage queue.
\item[keyOffset] Calculates the tone that is equivalent to zero in Finale's
coding. This depends on the active key signature.
\item[toneName] Converts a Finale code for tone height into a string, with
respect to accidentals and flag-rests.
\item[ksName] Converts a key signature into a GUIDO tag. Major/minor information
is preserved.
\item[durName] Converts a duration to GUIDO. 4096 is equivalent to a whole note,
1024 to a quarter, and so on.
\item[getTimeSig] Reads the time signature from Finales data structures,
converting it to a fraction.
\item[accumulateTS] Calculates a fraction describing the total elapsed time
up to the current measure. This is done by summing up all the time signatures.
\item[underfullMeasure] Builds a string that contains an appropriate
\verb"empty" or \verb"_" for a gap-rest at the end of a measure.
\item[LIFSET] expands to an \verb"if" clause that checks if an option bit is set.
\item[LIFNOT] Similarly, checks if the bit is not set.
\item[LEND] expands to \verb"}". This is for readability only.
\item[readMusic] gets a destination file name and then iterates through
Finale's data structures, generating GUIDO code and writing it to that file.\\
The loops, from outer to inner:
\begin{itemize}
\item slot (staff)
\item layer
\item measure
\item entry (chord)
\item note
\end{itemize}
\item[f2gDHandler] processes messages sent to the dialog window, setting
\verb"gmnFormat" accordingly.
\end{description}

\section{Finale's data structures}
I have not yet found any nearly complete documentation about this topic. So this is a
rather short section.\\
There is a database for measure-wise and global information. To use it, one needs
to know the name of a function that loads the desired piece of information, as well
as correct values for two indices called \verb"cmper" and \verb"inci". \verb"cmper"
is often a measure number, and \verb"inci" is sometimes unused.
An example of using these can be found in \verb"getTimeSig".\\
Most other (thus fine-grain) information is stored in the EXTGF structure and
dependent structures.

\section{Known bugs}
\begin{itemize}
\item Beaming doesn't work correctly in complicated cases.
\item Some unrecognized input may be ignored without message.
\end{itemize}

\section{Ideas for improvement}
The most important task will be the drawing of a map of Finale's data structures, I guess. As there is now
a running skeleton plugin, understanding Finale's internal music representation and searching for related
information will become far more important.\\
As you might have noticed, \verb"readMusic" has become large, but my first attempts to break it down made it even
less readable. A re-organisation might make (bug-free) extensions to the plugin easier.\\
It would also be possible to introduce an internal representation for music notation to the plugin, achieving
a much more elegant architectural design - although I don't think this is really necessary. If one wants to do
so, it's probably easiest to take this plugin's source as an example and start from scratch.\\
Last (and probably least) the current user interface is less than optimal and could benefit from some attention.

\end{document}






