\documentclass{article}
\usepackage{jim,amsmath}
\usepackage[utf8]{inputenc}
%\usepackage[francais]{babel}
\usepackage[T1]{fontenc}
%\usepackage{pxfonts}
\usepackage{graphicx}
\usepackage{verbatim}
\usepackage{cprotect}
\usepackage{hyperref}

\usepackage{natbib}

\newcommand{\verburl}[1]{
\begin{quote}
\begingroup
\fontsize{7.5pt}{12pt}\selectfont
#1
\endgroup
\end{quote}
}
\newcommand{\guidosize}{6pt}

\title{The Guido HTTPD Server}
\twoauthors
  {Mike SOLOMON} {Grame \\ mike@mikesolomon.org}
  {Dominique FOBER} {Grame \\ fober@grame.fr}

\date{}
\begin{document}
\maketitle

\begin{abstract}
The Guido HTTPD Server is a RESTful server that exposes many elements of the public GUIDO Engine API to clients.  This article resumes the core tenants of the REST architecture and goes on to explain the functioning of the GUIDO server with numerous examples, concluding with potential applications of the server.
\end{abstract}


%---------------------------------------
\section{Introduction}
As client-server models for data processing, visualizing and analyzing become more widespread in mobile computing (YouTube, Instagram, SoundCloud), the music engraving community has yet to implement a client-server model that rapidly executes the compilation of musical scores and exposes an API allowing one to process and retrieve information about these scores.  Several existing tools (LilyBin, WebLily) are limited in the use of separate server-side executables to process user information.  The GUIDO note server (\url{http://www.noteserver.org/noteserver.html}) compiles GUIDO scores into images but does not aim to expose GUIDO's API to users.  To remedy this, a RESTful server has been implemented that stores GUIDO Music Notation Format and responds to queries about this notation.  By accessing the GUIDO Engine API directly, it is able to be rapid and informative without retaining information about the client's state.  After describing the core tenants of the RESTful architecture upon which the GUIDO HTTPD server is modeled, this article presents a detailed overview of the server, concluding with several application possibilities.

%---------------------------------------
\section{Representational state transfer}
The Representational state transfer (REST) standard is a server architecture style first elaborated in 2000 \cite{Fielding00}.  The REST architecture is elaborated as a response to existing hypermedia systems and is intended as a set of constraints to facilitate exchange in these systems.  The standard is based on a traditional client-server model with the design trade-off that the server is \emph{stateless}, meaning that all of the information required to process a request is contained in the request itself and the server does not need to store intermediary states.  In order to speed up interaction with the server, the REST architecture calls for client-side caching of data, which can potentially eliminate certain redundant server requests.  It also calls for a uniform interface, harmonizing all applications' interactions with the server at the expense of application-specific interaction models that could speed up exchanges.  Layering is possible in this model, with intermediary servers translating various forms of shorthand into standard server commands.  With this layering comes the constraint that exchanging agents cannot ``see'' beyond the layer with which they are communicating.  Like other aspects of REST, this is intended to encapsulate all information in a single request made to a single agent.  As the burden on the client to be server-compliant is high in REST, the standard provides an optional constraint of code-on-demand to be downloaded from the server (scripts, applets, etc.) to ease client-side software development.\par
Certain specific architectural elements are put into place in order to facilitate the above-described architecture  In addition to the transferring of data, REST calls for the transferring of meta-data about a server response.  This allows for the client side to have information about how to de-encode the response without needing to send specific de-encoding instructions.  REST encourages resource requests that favor the retrieval of conceptually relevant entities rather than specific entities at a point in time.  For example, a request to the server for ``best movies of 1964'' or ``weather in Lyon'' should change with time rather than pointing to an arbitrary entity responding to this request at a given time.  Requests to the server should also specify the nature of the representation of these responses.\par
A server compliant with the REST architecture is said to be a RESTful server.

%---------------------------------------
\section{The GUIDO HTTPD server}
The GUIDO Hypertext Transfer Protocol Daemon (HTTPD) server is an austerely RESTful server that compiles strings written in the GUIDO Music Notation (GMN) Format and reports to the client several representations of this data.  It accepts user requests via two main methods of the HTTP protocol: GET, used to retrieve information about elements on the server, and POST, used to place elements on the server.
\subsection{The POST method}
POST, as implemented by the GUIDO server, is RESTful insofar as it does not save any information about the user state and only saves information sent by the user.\par
Assuming that a GUIDO HTTPD server is running on the subdomain \break \verb=http://guido.grame.fr= on port \verb=8000=, a POST request containing GMN code \verb=[a b c d]= is sent via \verb=curl= as follows:
\begin{quote}
\begingroup
\fontsize{\guidosize}{12pt}\selectfont
\begin{verbatim}
curl -d"data=[a b c d]" http://guido.grame.fr:8000
\end{verbatim}
\endgroup
\end{quote}
%\verburl{curl -d"data=[a b c d]" http://guido.grame.fr:8000}
Assuming that the GMN code is valid, response, in JSON, gives the user a unique identifier generated using an SHA-1 tag corresponding to the input file.  This ensures that the server will not compile and store the same information multiple times:
\begin{quote}
\begingroup
\fontsize{\guidosize}{12pt}\selectfont
\begin{verbatim}
{
  "ID": "07a21ccbfe7fe453462fee9a86bc806c8950423f"
}
\end{verbatim}
\endgroup
\end{quote}
This is the server's internal representation of the GMN code and used for all subsequent requests to the server.  To access it, it is appended onto the URI.  The following is a simple request using the SHA-1 tag:
\begin{quote}
\begingroup
\fontsize{\guidosize}{12pt}\selectfont
\begin{verbatim}
curl http://guido.grame.fr:8000/
07a21ccbfe7fe453462fee9a86bc806c8950423f
\end{verbatim}
\endgroup
\end{quote}
It results in the image seen in Figure~\ref{fig:figure1}.
\begin{figure}[h]
  \centering
    \includegraphics[width=0.5\textwidth]{figure1}
  \cprotect\caption{\label{fig:figure1}Score with SHA-1 tag \verb=07a21ccbfe7fe453462fee9a86bc806c8950423f=.}
\end{figure}


\subsection{The GET method}
True to RESTful form, the server does not store internally any information about user-requests sent via the GET method save an optional Apache-style log file with various levels of detail possible.  The main return type is JSON for all queries related to information about a score, MIDI for midi realizations of the score, and PNG for all queries asking for visual representations of the score itself.  The latter is also possible in JPEG and SVG.\par
The GUIDO HTTPD server attempts to expose as much of the public API of the GUIDO Engine as possible, focussing on one-to-one equivalencies with its functions when possible.  Arguments are passed to these functions via optional key-value pairs in the URI.  Defaults are provided for all key-value pairs in case of omission.  All commands are appended to a unique SHA-1 identifier for uploaded GMN code. An exhaustive overview of the API can be found in the GUIDO HTTPD server's documentation\cite{guidoserverdoc}.\par
This section aims to discuss some of the broad decisions made in exposing a C API via a web interface, giving three exhaustive examples at the end showing how the API is exposed.

\subsubsection{Function as URI element}
A function in the GUIDO public API is represented as an element of the URI sent to the server.  For example, the function \verb=GuidoGetPageCount= in the GUIDO public API is represented as the URI element \verb=pagescount=.\par
The GUIDO public API provides two generic categories of functions:
\begin{itemize}
\item Functions reporting information about GUIDO
\item Functions reporting information about a specific score processed by GUIDO.
\end{itemize}
To represent this distinction via that GUIDO server, the function is elaborated as a URI element either after the server's URI (for information about GUIDO) or after the SHA-1 tag of a score (for information about a specific score processed by GUIDO).  For example,
\begin{quote}
\begingroup
\fontsize{\guidosize}{12pt}\selectfont
\begin{verbatim}
curl http://guido.grame.fr:8000/version
\end{verbatim}
\endgroup
\end{quote}
reports the version of both GUIDO and the GUIDO server and thus does not need a SHA-1 tag.  The URI
\begin{quote}
\begingroup
\fontsize{\guidosize}{12pt}\selectfont
\begin{verbatim}
curl http://guido.grame.fr:8000/
07a21ccbfe7fe453462fee9a86bc806c8950423f/voicescount
\end{verbatim}
\endgroup
\end{quote}
exposes the API function \verb=GuidoCountVoices= via the URI element \verb=voicescount=, giving the voice count of specific score.  Below is a succinct list of the servers' naming conventions showing the name of a function in the GUIDO public API, its representation as a server URI element, and if it is score-specific or generic to all of GUIDO.
\begin{table}
\begin{tabular}{|l|l|l|}\hline
API Function & URI Element & score? \\\hline
\verb=GuidoGetPageCount= & \verb=pagescount= & Yes \\\hline
\verb=GuidoGetVoiceCount= & \verb=voicescount= & Yes \\\hline
\verb=GuidoDuration= & \verb=duration= & Yes \\\hline
\verb=GuidoFindPageAt= & \verb=pageat= & Yes \\\hline
\verb=GuidoGetPageDate= & \verb=pagedate= & Yes \\\hline
\verb=GuidoGetPageMap= & \verb=pagemap= & Yes \\\hline
\verb=GuidoGetSystemMap= & \verb=systemmap= & Yes \\\hline
\verb=GuidoGetStaffMap= & \verb=staffmap= & Yes \\\hline
\verb=GuidoGetVoiceMap= & \verb=voicemap= & Yes \\\hline
\verb=GuidoGetTimeMap= & \verb=timemap= & Yes \\\hline
\verb=GuidoAR2MIDIFile= & \verb=midi= & Yes \\\hline
\verb=GuidoGetVersionStr= & \verb=version= & No \\\hline
\verb=GuidoGetLineSpace= & \verb=linespace= & No \\\hline
\end{tabular}
\end{table}
Note that the only generic URI element that does not correspond to a GUIDO public API function is ``server``, which gives the version number of the server and thus is not related to the GUIDO API proper.
\subsubsection{Arguments as key-value pairs}
Several of the API functions provided above require arguments in order to provide results.  For example, the function \verb=GuidoGetStaffMap= requires an argument \verb=staff= specifying the staff for which the map should be generated.  These arguments are specified in key-value pairs in the URI.
\begin{quote}
\begingroup
\fontsize{\guidosize}{12pt}\selectfont
\begin{verbatim}
curl http://guido.grame.fr:8000/
07a21ccbfe7fe453462fee9a86bc806c8950423f/staffmap?staff=1
\end{verbatim}
\endgroup
\end{quote}
Default arguments are provided for all argument-taking functions in case the user fails to specify an argument.\par
With respect to the images being generated from the server, the GUDIO server allows for the specification of several parameters relating to the layout and formatting of scores as key-value pairs.  These parameters are represented in several different ways in the GUIDO public API.  Some, such as \verb=topmargin=, are become values of structures such as \verb=GuidoPageFormat=.  Others, such as \verb=resize=, represent calls to functions that effect layout (in this case \verb=GuidoResizePageToMusic=).  Yet others, such as \verb=width=, are used at several points in the layout process depending on the chosen backend.  Rather than being consistent in the mapping of the API to a URI convention, we have implemented all layout and formatting options as key-value pairs to make interacting with the server easier.\par
In order to deal with malformed URIs, the GUIDO server provides defaults for out-of-range or nonsensical values and ignores nonsensical keys.
\begin{quote}
\begingroup
\fontsize{\guidosize}{12pt}\selectfont
\begin{verbatim}
curl http://guido.grame.fr:8000/
07a21ccbfe7fe453462fee9a86bc806c8950423f/?
topmargin=0.0&format=jpg&resize=useless&foo=bar
\end{verbatim}
\endgroup
\end{quote}
In the above example, the \verb=topmargin= and \verb=format= values are not used whereas \verb=resize= is given its default value because \verb=useless= is  useless and \verb=foo= is ignored because it is not a valid key.%\par
\begin{comment}
The following table summarizes the various formatting and layout options one can provide to the server along with the current defaults and their units.  Note that the GUIDO unit represents an arbitrary internal unit that must be experimented with to fine tune.\par
\begin{table}
\begin{tabular}{|l|p{4cm}|l|l|}\hline
Value & Role & Default & Unit \\\hline
\verb=page= & a page number & 1 & page \\\hline
\verb=width= & the drawing area width & 15 & cm \\\hline
\verb=height= & the drawing area height & 5 & cm \\\hline
\verb=marginleft= & the left margin & 1 & cm \\\hline
\verb=marginright= & the right margin & 1 & cm \\\hline
\verb=margintop= & the top margin & 0.5 & cm \\\hline
\verb=marginbottom= & the bottom margin & 0.5 & cm \\\hline
\verb=format= & image format & png & png/jpg/svg \\\hline
\verb=resize= & resize the page to the image & yes & yes/no \\\hline
\verb=sysDistance= & distance between systems & 75 & GUIDO \\\hline
\verb=sysDistribution= & controls systems distribution & auto & auto/always/never\\\hline
\verb=sysDistribLimit= & maximum distance allowed between two systems for automatic distribution mode & 0.25 & fraction of page \\\hline
\verb=force= & force in the Space-Force function & 750.0 & GUIDO \\\hline
\verb=spring= & spring stiffness & 1.1 & GUIDO \\\hline
\verb=neighborhoodSpacing= & should we use the neighborhood spacing algorithm? & no & yes/no \\\hline
\verb=optimalPageFill= & should we use the optimal page fill algorithm & yes & yes/no\\\hline
\end{tabular}
\end{table}
\end{comment}
\subsubsection{Example: voicescount}
The command voicescount returns the number of voices in a score.  It exposes the GUIDO Engine API method \verb=GuidoCountVoices=.  For example, the request:
\begin{quote}
\begingroup
\fontsize{\guidosize}{12pt}\selectfont
\begin{verbatim}
curl http://guido.grame.fr:8000/
07a21ccbfe7fe453462fee9a86bc806c8950423f/voicescount
\end{verbatim}
\endgroup
\end{quote}
yields the following result:
\begin{quote}
\begingroup
\fontsize{\guidosize}{12pt}\selectfont
\begin{verbatim}
{
  "07a21ccbfe7fe453462fee9a86bc806c8950423f": {
    "voicescount": 1
  }
}
\end{verbatim}
\endgroup
\end{quote}


\subsubsection{Example: pageat}
The command pageat returns the page given a specific date, expressed as a rational number.  It exposes the GUIDO Engine API method \verb=GuidoFindPageAt=.  For example, the request:
\begin{quote}
\begingroup
\fontsize{\guidosize}{12pt}\selectfont
\begin{verbatim}
curl http://guido.grame.fr:8000/
07a21ccbfe7fe453462fee9a86bc806c8950423f/pageat?date=1/4
\end{verbatim}
\endgroup
\end{quote}
yields the following result:
\begin{quote}
\begingroup
\fontsize{\guidosize}{12pt}\selectfont
\begin{verbatim}
{
  "07a21ccbfe7fe453462fee9a86bc806c8950423f": {
    "page": 1,
    "date": "1/4"
  }
}
\end{verbatim}
\endgroup
\end{quote}

\subsubsection{Example: staffmap}
The command staffmap returns a map of the space each element of a given staff takes up in 2D space (represented by a box) and time space (represented as an interval of rational numbers).  It exposes the GUIDO Engine API method \verb=GuidoGetStaffMap=.  For example, the request:
\begin{quote}
\begingroup
\fontsize{\guidosize}{12pt}\selectfont
\begin{verbatim}
curl http://guido.grame.fr:8000/
07a21ccbfe7fe453462fee9a86bc806c8950423f/staffmap?staff=1
\end{verbatim}
\endgroup
\end{quote}
yields the following result:
\begin{quote}
\begingroup
\fontsize{\guidosize}{12pt}\selectfont
\begin{verbatim}
{
  "07a21ccbfe7fe453462fee9a86bc806c8950423f": {
    "staffmap": [
      {
        "graph": {
          "left": 916.18,
          "top": 497.803,
          "right": 1323.23,
          "bottom": 838.64
        },
        "time": {
          "start": "\"0/1\"",
          "end": "\"1/4\""
        }
      },
      {
        "graph": {
          "left": 1323.23,
          "top": 497.803,
          "right": 1730.28,
          "bottom": 838.64
        },
        "time": {
          "start": "\"1/4\"",
          "end": "\"1/2\""
        }
      },
      {
        "graph": {
          "left": 1730.28,
          "top": 497.803,
          "right": 2137.33,
          "bottom": 838.64
        },
        "time": {
          "start": "\"1/2\"",
          "end": "\"3/4\""
        }
      },
      {
        "graph": {
          "left": 2137.33,
          "top": 497.803,
          "right": 2595.51,
          "bottom": 838.64
        },
        "time": {
          "start": "\"3/4\"",
          "end": "\"1/1\""
        }
      }
    ]
  }
}
\end{verbatim}
\endgroup
\end{quote}

\section{Applications}
In keeping with the REST architecture, the GUIDO server is application agnostic, favoring generality over specificity.  As a result, it is likely ill-suited for direct end-user queries and can instead be buffered by intermediary layers used to interact with users.  As mobile music making becomes increasingly common, the server is intended as an alternative to embarking libGUIDOEngine with mobile applications.  In addition to shielding developers and client-side users from build problems with mobile platform updates, outsourcing to a server allows multiple projects to mutualize a rendering engine and share information about usage.\par
The process of elaborating the GUIDO server as a RESTful server is also a first step towards a more robust project to implement a wiki-like crowd music engraving platform that uses at its core a restful server storing data in various domain-specific languages.

\nocite{*}
\bibliographystyle{plain}
\bibliography{../../guido}
\end{document}